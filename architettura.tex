\documentclass[border=15pt, multi, tikz]{standalone}
\usepackage{import}
\subimport{./layers/}{init}
\usetikzlibrary{positioning}

\def\FcColor{rgb:blue,5;red,2.5;white,5}
\def\FcReluColor{rgb:blue,5;red,5;white,4}
\def\ConvColor{rgb:yellow,5;red,2.5;white,5}
\def\ConvReluColor{rgb:yellow,5;red,5;white,5}
\def\PoolColor{rgb:red,1;black,0.3}
\def\AvgPoolColor{rgb:red,1;black,0.5}
\def\DcnvColor{rgb:blue,5;green,2.5;white,5}
\def\SoftmaxColor{rgb:magenta,5;black,7}
\def\SumColor{rgb:blue,5;green,15}
\def\poolsep{0.5}
\def\poolwidth{0.4}
\def\fcsep{1}
\def\FlattenColor{rgb:black,0.5}



\begin{document}
\begin{tikzpicture}

%%% REDEFINING CONNECTION ARROW STYLE AND WIDTH (./layers/init.tex) %%%
\def\edgecolor{rgb:black,1}
\newcommand{\midarrow}{\tikz \draw[-Stealth,line width =0.3mm,draw=\edgecolor] (-0.3,0) -- ++(0.3,0);}
\tikzstyle{connection}=[thick,every node/.style={sloped,allow upside down},draw=\edgecolor,opacity=0.6]


%%% CONVOLUTIONAL BLOCKS %%%
\pic[shift={(0,0,0)}] at (0,0,0) {RightBandedBox={name=in1, caption=input,%
        xlabel={{"15","15"}},zlabel=20,fill=\ConvColor,bandfill=\ConvReluColor,%
        height=2, width=2, depth=20}};

\pic[shift={(1,0,0)}] at (0,0,0) {RightBandedBox={name=cr1, caption=conv1, %
        xlabel={{"115","115"}},zlabel=20,fill=\ConvColor,bandfill=\ConvReluColor,%
        height=2, width={3,3}, depth=20}};
\pic[shift={(0,0,0)}] at (cr1-east) {Box={name=p1,%
        fill=\PoolColor,opacity=0.5,height=2,width=\poolwidth,depth=10}};

\pic[shift={(\poolsep,0,0)}] at (p1-east) {RightBandedBox={name=cr2, caption=conv2,%
        xlabel={{"230","230"}},zlabel=10,fill=\ConvColor,bandfill=\ConvReluColor,%
        height=2, width=4, depth=10}};
\pic[shift={(0,0,0)}] at (cr2-east) {Box={name=p2,%
        fill=\PoolColor,opacity=0.5,height=2,width=\poolwidth,depth=5}};

\pic[shift={(\poolsep,0,0)}] at (p2-east) {RightBandedBox={name=cr3, caption=conv3,%
        xlabel={{"460","460"}},zlabel=5,fill=\ConvColor,bandfill=\ConvReluColor,%
        height=2, width=5, depth=5}};
\pic[shift={(0,0,0)}] at (cr3-east) {Box={name=p3,%
        fill=\PoolColor,opacity=0.5,height=2,width=\poolwidth,depth=2}};

%%%%% NON-POOLING LAYERS %%%%%
\pic[shift={(\poolsep,0,0)}] at (p3-east) {RightBandedBox={name=cr4, caption=conv4,%
        xlabel={{"460","460","460","920"}},zlabel=2,fill=\ConvColor,bandfill=\ConvReluColor,%
        height=2, width={5,5,5,7}, depth=2}};
\pic[shift={(0,0,0)}] at (cr4-east) {Box={name=p4,%
        fill=\AvgPoolColor,opacity=0.5,height=2,width=\poolwidth,depth=1}};


%%% FULLY CONNECTED LAYERS %%%
\pic[shift={(\fcsep,0,0)}] at (p4-east) {RightBandedBox={name=fc1, caption=fc1,%
        xlabel=1, zlabel=64,fill=\FcColor, bandfill=\FcReluColor, opacity=0.5,%
        height=2, width=2, depth=100}};
\pic[shift={(\fcsep,0,0)}] at (fc1-east) {RightBandedBox={name=fc2, caption=fc2,%
        xlabel=1, zlabel=128,fill=\FcColor, bandfill=\FcReluColor,opacity=0.5,%
        height=2, width=2, depth=120}};
\pic[shift={(\fcsep,0,0)}] at (fc2-east) {Box={name=fc3, caption=output,%
        xlabel=1, zlabel=2,fill=\FcColor, opacity=0.5,%
        height=2, width=2, depth=6}};

%%% FLATTENING LINES TO FC LAYER %%%
\foreach \x in {5,10,...,50}
  \draw[dashed, opacity=0.3]
    (fc1-west)++(0, 1.0*.2, \x*.2) -- (p4-neareast)
    (fc1-west)++(0, 1.0*.2, -\x*.2) -- (p4-fareast);

% %%% FLATTENING LINES TO FC LAYER %%%
% \foreach \x in {5,10,...,60}
%   \draw[dashed, opacity=0.3]
%     (fc2-west)++(0, 1.0*.2, \x*.2) -- (fc1-southwest)
%     (fc2-west)++(0, 1.0*.2, -\x*.2) -- (fc1-southwest);

% %%%%%%%%%%%%%%%%%%%%%%%%%%%%%%%%%%%%%%%%%%%%%%%%%%%%%%%%%%%%%%%%%%%%%%%%%%%%%%%%%%%%%%%%%
% %%% Draw connections
% %%%%%%%%%%%%%%%%%%%%%%%%%%%%%%%%%%%%%%%%%%%%%%%%%%%%%%%%%%%%%%%%%%%%%%%%%%%%%%%%%%%%%%%%%
\draw [connection]  (in1-east)    -- node {\midarrow} (cr1-west);
%\draw [connection]  (cr1-east)    -- node {\midarrow} (p1-west);
\draw [connection]  (p1-east)    -- node {\midarrow} (cr2-west);
%\draw [connection]  (cr2-east)    -- node {\midarrow} (p2-west);
\draw [connection]  (p2-east)    -- node {\midarrow} (cr3-west);
%\draw [connection]  (cr3-east)    -- node {\midarrow} (p3-west);

%%% SECOND LINE CONNECTIONS %%%
\draw [connection]  (p3-east)    -- node {\midarrow} (cr4-west);
%\draw [connection]  (cr4-east)    -- node {\midarrow} (p4-west);
 \draw [connection]  (p4-east)    -- node {\midarrow} (fc1-west);
% \draw [connection]  (p3-east)    -- node {\midarrow} (cr4-west);
 \draw [connection]  (fc1-east)    -- node {\midarrow} (fc2-west);
 \draw [connection]  (fc2-east)    -- node {\midarrow} (fc3-west);

\end{tikzpicture}
\end{document}\grid
