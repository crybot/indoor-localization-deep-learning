\documentclass{standalone}
\begin{document}
%Problema
\section{Localizzazione Indoor}
Il problema della Localizzazione Indoor consiste nell'individuazione di un
utente all'interno di uno spazio chiuso e in riferimento a un sistema di
coordinate predefinito. Tale sistema di coordinate, relativo ad un determinato
edificio, può essere poi espresso in termini georeferenziali conoscendo la
precisa dislocazione geografica del locale in questione. \\
La localizzazione indoor apre le porte a diverse possibilità nel campo
dell'esperienza utente all'interno di edifici pubblici, nel settore
della gestione dei flussi di persone, della sicurezza e della contingentazione.
Attraverso l'impiego di tale tecnologia è possibile coadiuvare la navigazione
degli utenti all'interno di edifici complessi e migliorare l'esperienza
individuale di persone affette da disabilità. Per ottenere questi risultati è
però richiesto un certo grado di precisione, di affidabilità, di efficienza e
di sicurezza nella gestione della privacy dei dati di localizzazione degli
utenti. Inoltre la tecnologia scelta per risolvere il problema, per essere
fruibile, deve avere come ulteriore requisito il basso impatto economico.

\section{Soluzioni Tecnologiche}
%Letteratura
Nel corso degli anni sono state implementati diversi sistemi di
localizzazione indoor, che possiamo dividere in due macrocategorie: soluzioni
ad-hoc e soluzioni che sfruttano tecnologie esistenti. Nel primo caso si
fornisce all'utente l'attrezzatura necessaria ad essere localizzato, mentre nel
secondo si utilizza un dispositivo mobile di proprietà dell'utilizzatore.
Spesso tale dispositivo è uno smartphone. \\ I sistemi che implementano
tecnologie sviluppate ad-hoc, sono spesso più efficienti, più precisi e
flessibili. Tuttavia il loro impiego rimane limitato dall'alto costo di
progettazione, installazione e di gestione. È poi richiesto che ad ogni utente
che intende essere localizzato sia assegnato un dispositivo che si interfacci
col sistema impiegato. \\
Per l'impiego su larga scala, un sistema di localizzazione indoor deve essere
facilmente utilizzabile dalle masse e non deve richiedere particolari requisiti
tecnologici.  \\

{\LARGE TODO: Inserire riferimenti bibliografici che mettano in comparazione le
  varie tecnologie utilizzate, ad-hoc e non, e in particolare mostrino i
  risultati dei sistemi che sfruttano il Bluetooth.}

\section{Bluetooth Low Energy}
La tecnologia \emph{Bluetooth} è talmente pervasiva che
ogni smartphone in circolazione ne implementa il protocollo, mostrandosi
particolarmente adeguata alla risoluzione del problema in esame. Nello
specifico, \emph{Bluetooth Low Energy} (BLE) è un protocollo che riduce
notevolmente il consumo energetico dei dispositivi che ne sfruttano le
capacità. \\
La soluzione riportata in questo documento prevede l'utilizzo di una serie di
beacon BLE programmabili, ciascuno installato in un punto significativo
dell'edificio e configurato per emettere un segnale broadcast con una frequenza
di circa 50Hz. La potenza dei segnali viene quindi utilizzata per produrre,
attraverso l'utilizzo di una rete neurale artificiale, una coppia di coordinate
rappresentative della posizione dell'utente all'interno dell'edificio. Ciò
viene reso possibile da una fase preliminare in cui viene mappata la superficie
del locale raccogliendo i segnali ricevuti dai beacon in vari punti. Per ogni
punto della superficie mappato si registra una serie temporale di segnali, dei
quali si considera solo il valore \emph{RSSI}, ovvero la potenza del segnale
nel punto in cui questo viene ricevuto. \\
Il modello utilizzato è di fatto completamente agnostico rispetto
all'ubicazione dei beacon installati, fin quando questa sia unica e non mutata
nel tempo. \\
L'utilizzo di tale sistema assicura il completo anonimato dell'utente, il quale
non necessita di condividere la propria posizione, essendo quest'ultima
calcolata direttamente sul suo smartphone in funzione dei segnali che riceve.

% Contenuto
Questa tesi si pone l'obiettivo di descrivere nello specifico la rete neurale
progettata per risolvere il problema, le tecniche utilizzate per alzare il
grado di precisione del modello e le principali differenze rispetto a modelli
già esistenti.


\end{document}
